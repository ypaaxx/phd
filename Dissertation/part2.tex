\setcounter{chapter}{1}
\chapter{Расчётное исследование течения в проточной части вентилятора с меридиональным ускорением потока}

Течение в вентиляторе с меридиональным ускорением потока имеет ряд отличий от течения в вентиляторе с цилиндрической проточной частью. Отличия обусловлены наличием дополнительных параметров, влиянием которых на течение которых, принято пренебрегать. Искажение течения, при некоторых комбинациях дополнительных параметров, становится настолько сильным, что пренебрежение ими приводит к заметной ошибке в характеристиках проектируемой машины. 

В первую очередь требуется определить границу параметров, при которых пренебрежение характером течения приводит к ошибке. Для этого необходимо проанализировать большое количество вариантов геометрии проточных частей. Варьирование только одного из основных параметров, такого как угол между поверхностью втулки и осью машины или коэффициент ускорения меридиональной компоненты скорости, приводит к кратному увеличению количества вариантов.

Экспериментальное исследование такого большого числа моделей вентиляторов нереалестично. В то же время использование пересчёта данных с использованием конформных отображений приводит к искажению течения в пограничном слое. Эти обстоятельства исключают сугубо экспериментальных подход к данному исследованию.

Современные методы численного моделирования течения жидкости решением сеточных уравнений подходят для исследования характеристик машины с известной геометрией. Однако для анализа, требующего  большого количества возможных вариантов геометрии, являются слишком трудо- и времязатратными. При этом не обладающими высокой достоверностью. 
Инженерные методы проектирования, основанные на упрощённых моделях, в достаточной мере дают представление о течении и не требуют больших ресурсов.

\section{Деление течения на осесимметричные струйки тока}

Гипотеза плоских сечений, хорошо работающая для цилиндрической проточной части, не подходит для течения в пространственной решётке профилей. Однако деление сложной проточной части на более простые области, течение в которых не влияет на течения в соседних областях, сильно упрощает анализ. Это снижает количество возможных вариантов геометрии проточной части, так как новую геометрию возможно получить простым добавлением и удалением краевых областей. Так же, рассмотрение течения в отдельной области снижает количество варьируемых параметров.

\section{Математическое моделирование течения в межлопаточном канале}

	\subsection{Конформное отображение}

Одним из наиболее эффективных подходов является конформное отображение осесимметричной поверхности тока на плоскость, как это представлено на рисунке.
Рассмотрим элементарную струйку тока, заключенную между двумя бесконечно близкими поверхностями тока, следуя допущению, что поверхности тока осесимметричны, под толщиной струи \(h\) примем расстояние между ограничивающими поверхностями. Осесимметричная поверхность описывается некоторой зависимостью радиуса от осевой координаты \(r=r(z)\) и образуется поворотом линии вокруг продольной оси \(z\).

	\subsection{Потенциальное двухмерное течение в слое переменной толщины в движущейся решётке профилей}
	\subsection{Вычисление скорости жидкости}
	\subsection{Решение методом дискретных вихрей}
	\subsection{Сравнение результатов моделирований выполненных двумя методами}

\section{Повышение напора в рабочем колесе}
	\subsection{Исследуемые решётки профилей}
	\subsection{Повышение напора в рабочем колесе}
	\subsection{Сравнение результатов моделирований выполненных двумя методами}
	
\section{Влияние формы проточной части на меридиональные линии тока}
\section{Влияние угла наклона линии тока на течения в эквивалентных решётках}
\section{Влияние вращения решётки}

\FloatBarrier
