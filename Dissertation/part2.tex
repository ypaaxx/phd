\chapter{Расчётное исследование течения в проточной части вентилятора с меридиональным ускорением потока}

Стоит сказать, что на основании главы \ref{ch:ch1}, необходимо изучить это хитрое течение подробнее

\section{Математическое моделирование течения в межлопаточном канале}

	\subsection{Конформное отображение}

Одним из наиболее эффективных подходов является конформное отображение осесимметричной поверхности тока на плоскость, как это представлено на рисунке.
Рассмотрим элементарную струйку тока, заключенную между двумя бесконечно близкими поверхностями тока, следуя допущению, что поверхности тока осесимметричны, под толщиной струи \(h\) примем расстояние между ограничивающими поверхностями. Осесимметричная поверхность описывается некоторой зависимостью радиуса от осевой координаты \(r=r(z)\) и образуется поворотом линии вокруг продольной оси \(z\).

	\subsection{Конформное отображение}
	\subsection{Потенциальное двухмерное течение в слое переменной толщины в движущейся решётке профилей}
	\subsection{Вычисление скорости жидкости}
	\subsection{Решение методом дискретных вихрей}
	\subsection{Сравнение результатов моделирований выполненных двумя методами}

\section{Повышение напора в рабочем колесе}
	\subsection{Исследуемые решётки профилей}
	\subsection{Повышение напора в рабочем колесе}
	\subsection{Сравнение результатов моделирований выполненных двумя методами}
	
\section{Влияние формы проточной части на меридиональные линии тока}
\section{Влияние вращения решётки}

\FloatBarrier
