\chapter{Расчётное исследование течения в проточной части вентилятора с меридиональным ускорением потока}

\section{Математическое моделирование течения в межлопаточном канале }

Одним из наиболее эффективных подходов является конформное отображение осесимметричной поверхности тока на плоскость, как это представлено на рисунке.
Рассмотрим элементарную струйку тока, заключенную между двумя бесконечно близкими поверхностями тока, следуя допущению, что поверхности тока осесимметричны, под толщиной струи \(h\) примем расстояние между ограничивающими поверхностями. Осесимметричная поверхность описывается некоторой зависимостью радиуса от осевой координаты \(r=r(z)\) и образуется поворотом линии вокруг продольной оси \(z\).

\FloatBarrier
