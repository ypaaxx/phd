\chapter*{\nomname}                         % Заголовок
\addcontentsline{toc}{chapter}{Список сокращений и условных обозначений}    % Добавляем его в оглавление
\eqexplSetIntro{}

\begin{eqexpl}
\item{\(P_\nu\)} полное давление вентилятора, \(\si\pascal\);
\item{\(P_{s\nu}\)} статическое давление вентилятора, \(\si\pascal\);
\item{\(P_{d\nu}\)} динамическое давление вентилятора, \(\si\pascal\);
\item{\(N\)} потребляемая вентилятором мощность, \(\si\watt\);
\item{\(Q\)} производительность вентилятора, $\si\meter^3/\si\second$;
\item{\(\eta\)} коэффициента полезного действия; 
\item{\(Re\)} число Рейнольдса;
\item{\(M\)} число Маха;
\item{\(b\)} хорда лопатки, $\si\meter$;
\item{\(W\)} скорость потока в относительном движении, $\si\meter/\si\second$;
\item{\(\nu\)} кинематическая вязкость воздуха, $\si\meter^2/\si\second$;
\item{\(a_{\text{зв}}\)} скорость звука в воздухе, $\si\meter/\si\second$;
\item{\(\bar{H}\)} коэффициент напора;
\item{\(\bar{H}_{\text{т}}\)} коэффициент теоретического напора;
\item{\(\bar{c}_{a}\)} коэффициент расхода;
\item{\(U_\text{к}\)} окружная скорость периферии рабочего колеса, $\si\meter/\si\second$;
\item{\(F_{\text{ом}}\)} кольцевая площадь проточной части, \(\si{\meter}^2\);
\item{\(\rho\)} плотность воздуха, \(\si{\kilogram}/\si{\meter}^3\);
\item{\(\psi\)}	коэффициент давления;
\item{\(\psi_{\text{т}}\)} коэффициент теоретического давления;
\item{\(\varphi_{a}\)} коэффициент осевой скорости;
\item{\(\varphi\)} коэффициент производительности;
\item{\(\lambda\)} коэффициент мощности;
\item{\(F\)} характерная площадь, \(\si{\meter}^2\);
\item{\(D\)} диаметр рабочего колеса, \(\si\meter\);
\item{\(n\)} частота вращения, об/мин;
\item{\(D\)} диаметр рабочего колеса, м;
\item{\( L_w \)} мощность акустического излучения, Вт;
\item{$\delta_{2}$} толщина потери импульса, м;
\item{$D_\text{eq}$} фактор эквивалентной диффузорности;
\item{$\beta$} угол между скоростью потока в относительном движении и осью решетки, \(\si\degree\);
\item{$\tau$} густота решетки;
\item{$S$}параметр закрутки;
\item{\(G_\theta\)} поток момента количества движения, \(\si\kilogram \cdot \si\meter^2/\si\second^2\);
\item{\(G_x\)} поток количества движения, \(\si\kilogram \cdot \si\meter/\si\second^2\);
\item{\(R\)} радиус сопла, \(\si\meter\);
\item{$\bar{d}$} относительный диаметр;
\item{\(\Delta P\)} потери полного давления, Па;
\item{$\zeta$} коэффициент потери полного давления;
\item{\(\pi\)} степень повышения давления в ступени;
\item{\(G\)} массовый расход, \(\si\kilogram/\si\second\);
\item{\(AVDR\)} отношение осевых плотностей тока;
\item{\(m\)} коэффициент меридионального ускорения;
\item{\(U\)} окружная скорость колеса, \(\si\meter/\si\second\);
\item{\(C_{u}\)} окружная компонента скорости потока, \(\si\meter/\si\second\);
\item{\(\bar{c}_{u}\)} коэффициент окружной компоненты скорости потока;
\item{\(\bar{r}\)} приведённый радиус;
\item{\(R_\text{к}\)} радиус рабочего колеса, \(\si\meter\);
\item{$\alpha$} угол между скоростью потока в абсолютном движении и осью решетки, \(\si\degree\);
\item{\(C_{\text{u}}\)} скорость потока в относительном движении перед лопаточным венцом на среднем радиусе, $\si\meter/\si\second$;
\item{\(W_{\text{u}}\)} скорость потока в абсолютном движении перед лопаточным венцом на среднем радиусе, $\si\meter/\si\second$;
\item{\(c\)}  толщина профиля, максимальное расстояние между ближайшими точками корытца и спинки, \(\si\meter\); 
\item{\(f\)} изгиб профиля, максимальное расстояние от средней линии до хорды профиля, \(\si\meter\);
\item{\(t\)} шаг решётки, расстояние между соответствующими точками соседних профилей, \(\si\meter\);
\item{\(\bar{c}\)} относительная толщина профиля; 
\item{\(\bar{f}\)} относительный изгиб профиля;
\item{\(\tau\)} густота решётки;
\item{\(\theta_{\text{г}}\)} угол установки, \(\si\degree\); 
\item{\(\beta_\text{л}\)} углы между касательными к средней линии профиля и осью решётки в точках передней и задней кромки, \(\si\degree\);
\item{\(\upsilon\)} угол изгиба профиля, \(\si\degree\);
\item{\(i\)} угол атаки, \(\si\degree\);
\item{\(\bar{x}_f\)} относительное положение точки средней линии профиля с максимальным изгибом;
\item{\(A, B\)} коэффициенты решётки профилей;
\item{\(r, \phi\)} полярные координаты;

\end{eqexpl}

\section*{Сокращения и индексы}
\begin{eqexpl}

\item{ВНА} входной направляющий аппарат;
\item{РК, К} рабочее колесо;
\item{СА} спрямляющий аппарат;
\item{a} осевая компонента;
\item{u} окружная компонента;
\item{m} меридиональная компонента;
\item{ср} величина на среднем радиусе, средняя величина;
\item{0} сечение перед ВНА;
\item{1} сечение перед РК;
\item{2} сечение перед СА;
\item{3} сечение после СА;

\end{eqexpl}

\eqexplSetIntro{где}
\FloatBarrier