
{\actuality} 
Осевые вентиляторы относятся к осевым турбомашинам. Они находят применения во многих областях промышленности. С их помощью осуществляется искусственная вентиляция промышленных и бытовых помещений, шахт, тоннелей метро и тоннелей автодорог. Используются в системах охлаждения, обогрева, циркуляции и кондиционирования воздуха. Их применяют для создания тяги летательных и подводных аппаратов.

Разнообразие областей применения осевых вентиляторов обуславливает широкий диапазон параметров работы. Создаваемое давление изменяется от единиц до тысяч Паскалей, а расход воздуха --- от сотых долей до нескольких тысяч кубических метров в секунду. Диаметры рабочих колёс составляют от сантиметров до десятков метров. Для систематизации огромного количества разнообразных вентиляторов используют безразмерные величины критериев подобия.

Важным параметром турбомашины является окружная скорость периферии рабочего колеса. К её величине приводятся прочие параметры для получения безразмерных коэффициентов. Из-за требований к прочности и шуму окружная скорость осевых вентиляторов, как правило, не превышает 100 м/с. В осевых насосах, применяемых в качестве водомётного движителя, ее величина еще меньше (до 60 м/с) из-за возникновения кавитации. Ограничение окружной скорости приводит к необходимости выбора больших коэффициентов напора вентилятора или росту количества ступеней.  

Таким образом, увеличение коэффициента напора осевых турбомашин является важной задачей, которую необходимо решать при создании широкого круга воздухоперемещающих и водомётных систем различного назначения. Особенно важным является решение этой задачи для осевых турбомашин, применяемых в авиационной, космической и морской технике, наземном транспорте. А так же для вентиляторов, применяемых в проветривании шахт, метрополитенов, для дутьевых вентиляторов и дымососов энергоблоков тепловых электростанций. 

Подавляющее число современных осевых вентиляторов выполнено с цилиндрической проточной частью, в этом случае поверхности втулки и корпуса параллельны оси вращения и площадь проходного сечения не изменяется вдоль течения. Такая форма проточной части проста в производстве и удобна при эксплуатации. А так же хорошо соответствует принятым в методиках проектирования допущениям, которые позволили успешно создавать и использовать эффективно работающие в заданных режимах работы турбомашины. Создано много работ по рекомендации выбора режимов работы и распределению параметров по высоте проточной части ступени для минимизации потерь полного давления.

Существенная часть потерь полного давления в ступени вентилятора обусловлена вязкостью рабочего тела, появлению пограничного слоя течения, нарастанию его толщины и затем отрыву. Более густые лопаточные венцы могут обеспечить б\'ольшую работу, подведённую к рабочему телу, однако при этом увеличивается и смачиваемая поверхность, на которой нарастает пограничный слой, что приводит к росту потерь полного давления.  
При увеличении коэффициента напора вентилятора выше некоторого оптимального значения, рост потерь превосходит рост подведённой работы, что приводит к снижению энергетической эффективности, характеризуемой коэффициентом полезного действия (КПД). Притом типичные значения коэффициента напора для осевого вентилятора или компрессора на порядок ниже значений подобной величины для осевой турбины. Причиной такого существенного отличия является диффузорное течение в ступени, которому сопутствует положительный градиент давления, приводящий к существенному нарастанию толщины пограничного слоя, а затем и к его отрыву.

Существуют методы снижения влияния пограничного слоя на основное течение в проточной части вентилятора. Например, отсос рабочего тела из пограничного слоя, или наоборот, вдув дополнительного рабочего тела в пограничный слой, для затягивания отрыва потока и снижения потерь полного давления. Или использование многорядных решёток профилей, в которых однорядные решётки имеют перекрытие по фронту. Применение таких методов на практике конструктивно сложно осуществимо. Одним из относительно легко применимых методов является снижение диффузорности течения за счёт применения меридионального ускорения потока, то есть постепенного увеличения величины проекции скорости потока на меридиональную плоскость (проходящую через ось) по ходу течения рабочего тела.  

Снижение диффузорности приводит к снижению градиента статического давление и уменьшению потерь полного давления. Однако методы проектирования таких вентиляторных ступеней не развиты настолько же хорошо как методы проектирования ступеней с циллиндрической проточной частью. Основным препятствием при проектировании является наличие дополнительных параметров геометрии проточной части, не учитываемых при проектировании цилиндрических ступеней, таких как распределение площадей проходного сечения по длине проточной части и угол наклона к оси вращения линий, образующих проточную часть. Работ о влиянии дополнительных параметров на аэродинамические характеристики ступени в открытом доступе не так много. 

Сегодня развитие электронно вычислительных машин (ЭВМ) и прикладных программ для анализа вязкого течения методами вычислительной гидродинамики позволяют проводить численные эксперименты значительно ускоряющие разработку и снижающие влияние физического эксперимента на проектирование. Однако эти методы не предназначены для предварительного проектирования, так как необходимо анализировать большое количество возможных конфигураций, для чего больше подходят инженерные методы основанные на упрощённых математических моделях.

Для вентиляторов с цилиндрической проточной частью накоплено огромное количество экспериментальных данных, по которым отлажены математические модели течения и взаимного влияния отдельных параметров. Накопление подобного опыта для нового типа вентиляторов затруднительно и следует по возможности использовать уже известные для случая с цилиндрической проточной части данные с необходимым пересчётом параметров. Необходимо определить, что общего есть у этих типов вентиляторов и что различного, для внесения соответствующих модификаций в разработанные методы анализа течения в проточной части.

% {\progress}
% Этот раздел должен быть отдельным структурным элементом по
% ГОСТ, но он, как правило, включается в описание актуальности
% темы. Нужен он отдельным структурынм элемементом или нет ---
% смотрите другие диссертации вашего совета, скорее всего не нужен.

{\aim} данной работы является разработка метода проектирования лопаточного аппарата высоконапорного вентилятора с меридиональным ускорением потока.

Для~достижения поставленной цели необходимо было решить следующие {\tasks}:
\begin{itemize}[beginpenalty=10000] % https://tex.stackexchange.com/a/476052/104425
	\item Разработать метод моделирования течения в лопаточном венце вентилятора с меридиональным ускорением потока.
	\item Исследовать влияние меридионального ускорения потока на аэродинамические характеристики решёток профилей.
	\item Разработать метод профилирования лопаточных венцов вентилятора с меридиональным ускорением потока.
%	\item Исследовать, разработать, вычислить и~т.\:д. и~т.\:п.
\end{itemize}

{\novelty}
\begin{itemize}[beginpenalty=10000] % https://tex.stackexchange.com/a/476052/104425
	\item Применён метод дискретных вихрей для анализа течения на осесимметричных поверхностях тока в слое переменной толщины.
	\item Проведено исследование влияния осевого ускорения потока на аэродинамические характеристики плоских решёток профилей.
	\item Выполнено экспериментальное исследование вентилятора с меридиональным ускорением потока.
\end{itemize}

{\influence}
состоит в том, что результаты работы могут быть использованы в ходе предварительного проектирования проточной части вентилятора для существенного снижения диапазона варьируемых параметров при последующем применении методов с использованием средств вычислительной гидродинамики.

{\methods} В работе использовались экспериментальные и теоретические методы исследования. Эксперименты проводились на физических моделях вентиляторов. В качестве теоретических методов использовались методы математического моделирования течения жидкости. Результаты теоретических исследований сравнивались с результатами полученными экспериментально.

{\defpositions}
\begin{itemize}[beginpenalty=10000] % https://tex.stackexchange.com/a/476052/104425
	\item 
	Метод математического моделирования потока невязкой жидкости на осесимметричных поверхностях тока в слое переменной толщины, основанный на конформном отображении поверхности течения и методе дискретных вихрей. Существующие программы для расчёта характеристик плоских решёток основанные на методе дискретных вихрей могут быть легко модифицированы для использования нового метода. В отличие от популярных современных методов вычислительной гидродинамики, получение решения методом дискретных вихрей не требует построения сетки во всей исследуемой области течения и занимает гораздо меньше времени, так как задача сводится к решению системы линейных алгебраических уравнений с относительно небольшим количеством неизвестных, способы решения которых на ЭВМ в настоящее время хорошо развиты и не требуют большой вычислительной мощности.
	
	\item 
	Метод профилирования вращающихся и неподвижных лопаточных венцов вентилятора с меридиональным ускорением, позволяющий увеличить точность совпадения расчётных и действительных распределений параметров по высоте проточной части. Это позволяет уменьшить количество численных и физических экспериментов необходимых при доводке. Форма проточной части оказывает значительное влияние на распределение меридиональной компоненты скорости и форму поверхностей тока. Новый метод позволяет учитывать это влияние.
	
	\item 
	Оценка влияния меридионального ускорения на аэродинамические характеристики плоских решёток профилей. Для различных решёток, составленных из профиля С-4, определялся изгиб профиля и угол поворота потока, соответствующие началу отрыва потока от спинки профиля в области задней кромки. Варьировался угол потока перед решёткой, густота и угол атаки сохранялись постоянными. Для моделирования течения использовался новый метод. По полученному распределению скоростей на контуре профиля методом Трукенбродта оценивалось состояние пограничного слоя и предсказывалась точка отрыва. Для некоторых режимов работы результаты сравнивались с результатами, полученными моделированием вязкого несжимаемого газа в пакете Ansys CFX, и показали хорошее совпадение. В результате исследования было определено, что меридиональное ускорение приводит к затягиванию отрыва и возможности получить большие коэффициенты напора в рабочем колесе при безотрывном обтекании.
	
	\item 
	Результаты экспериментального исследования модели вентилятора с меридиональным ускорением потока спроектированного в качестве движителя. Вентилятор состоял из рабочего колеса и спрямляющего аппарата. На стенде были получены аэродинамическая характеристика и распределение скорости по высоте проточной части за рабочим колесом на различных режимах работы. Сравнение полученных данных с расчётными и полученными в результате моделирования методами вычислительной гидродинамики в пакете Ansys CFX показывает возможность проектирования новым методом и валидирует моделирование течения в CFX.
	
	\item 
	Меридиональное ускорение потока позволяет повысить КПД вентилятора в широкой области расчётных параметров. Проведено сравнение результатов численного моделирования характеристик вентиляторов с различной степенью меридионального ускорения потока спроектированных на одно задание. Моделирование проводилось средствами вычислительной гидродинамики в программном пакете Ansys CFX. Сравнивался КПД вентиляторов на расчётном режиме.
\end{itemize}

{\reliability} полученных результатов обеспечивается следующим:
\begin{itemize}[beginpenalty=10000]
	\item
	Применение для расчёта течения в межлопаточных каналах и определения теоретических характеристик решёток профилей численным методом дискретных вихрей обосновано методическими и теоретическими работами представителей научной школы С.М. Белоцерковского. Методы профилирования лопаточных венцов вентиляторов по теоретическим характеристикам решёток полностью оправдали себя в работах, проведённых в ЦАГИ.
	\item 
	Для экспериментально полученных данных достоверность обеспечивается соответствием  испытательного стенда и методики испытаний ГОСТу.
	\item
	Для результатов полученных методами вычислительной гидродинамики достоверность подтверждается совпадением аэродинамических характеристик, таких как расходно-напорные характеристики и поля скоростей в межвенцовых зазорах с данными полученными экспериментально на моделях вентиляторов.
\end{itemize}

{\probation}
Основные результаты работы докладывались~на:
\begin{itemize}
	\item XXXIII научно-технической конференции по аэродинамике;
	\item XV Всероссийской конференции молодых учёных и специалистов (с международным участием) <<Будущее машиностроения России>>.
\end{itemize}

{\contribution} Автор принимал активное участие \ldots

{\publications} Основные результаты по теме диссертации изложены
в~2~печатных статьях в журналах, рекомендованных ВАК.

\begin{comment}
\ifnumequal{\value{bibliosel}}{0}
{%%% Встроенная реализация с загрузкой файла через движок bibtex8. (При желании, внутри можно использовать обычные ссылки, наподобие `\cite{vakbib1,vakbib2}`).
	{\publications} Основные результаты по теме диссертации изложены
	в~2~печатных статьях в журналах, рекомендованных ВАК.
}%

{%%% Реализация пакетом biblatex через движок biber
	\begin{refsection}[bl-author, bl-registered]
		% Это refsection=1.
		% Процитированные здесь работы:
		%  * подсчитываются, для автоматического составления фразы "Основные результаты ..."
		%  * попадают в авторскую библиографию, при usefootcite==0 и стиле `\insertbiblioauthor` или `\insertbiblioauthorgrouped`
		%  * нумеруются там в зависимости от порядка команд `\printbibliography` в этом разделе.
		%  * при использовании `\insertbiblioauthorgrouped`, порядок команд `\printbibliography` в нём должен быть тем же (см. biblio/biblatex.tex)
		%
		% Невидимый библиографический список для подсчёта количества публикаций:
		\printbibliography[heading=nobibheading, section=1, env=countauthorvak,          keyword=biblioauthorvak]%
		\printbibliography[heading=nobibheading, section=1, env=countauthorwos,          keyword=biblioauthorwos]%
		\printbibliography[heading=nobibheading, section=1, env=countauthorscopus,       keyword=biblioauthorscopus]%
		\printbibliography[heading=nobibheading, section=1, env=countauthorconf,         keyword=biblioauthorconf]%
		\printbibliography[heading=nobibheading, section=1, env=countauthorother,        keyword=biblioauthorother]%
		\printbibliography[heading=nobibheading, section=1, env=countregistered,         keyword=biblioregistered]%
		\printbibliography[heading=nobibheading, section=1, env=countauthorpatent,       keyword=biblioauthorpatent]%
		\printbibliography[heading=nobibheading, section=1, env=countauthorprogram,      keyword=biblioauthorprogram]%
		\printbibliography[heading=nobibheading, section=1, env=countauthor,             keyword=biblioauthor]%
		\printbibliography[heading=nobibheading, section=1, env=countauthorvakscopuswos, filter=vakscopuswos]%
		\printbibliography[heading=nobibheading, section=1, env=countauthorscopuswos,    filter=scopuswos]%
		%
		\nocite{*}%
		%
		{\publications} Основные результаты по теме диссертации изложены в~\arabic{citeauthor}~печатных изданиях,
		\arabic{citeauthorvak} из которых изданы в журналах, рекомендованных ВАК\sloppy%
		\ifnum \value{citeauthorscopuswos}>0%
		, \arabic{citeauthorscopuswos} "--- в~периодических научных журналах, индексируемых Web of~Science и Scopus\sloppy%
		\fi%
		\ifnum \value{citeauthorconf}>0%
		, \arabic{citeauthorconf} "--- в~тезисах докладов.
		\else%
		.
		\fi%
		\ifnum \value{citeregistered}=1%
		\ifnum \value{citeauthorpatent}=1%
		Зарегистрирован \arabic{citeauthorpatent} патент.
		\fi%
		\ifnum \value{citeauthorprogram}=1%
		Зарегистрирована \arabic{citeauthorprogram} программа для ЭВМ.
		\fi%
		\fi%
		\ifnum \value{citeregistered}>1%
		Зарегистрированы\ %
		\ifnum \value{citeauthorpatent}>0%
		\formbytotal{citeauthorpatent}{патент}{}{а}{}\sloppy%
		\ifnum \value{citeauthorprogram}=0 . \else \ и~\fi%
		\fi%
		\ifnum \value{citeauthorprogram}>0%
		\formbytotal{citeauthorprogram}{программ}{а}{ы}{} для ЭВМ.
		\fi%
		\fi%
		% К публикациям, в которых излагаются основные научные результаты диссертации на соискание учёной
		% степени, в рецензируемых изданиях приравниваются патенты на изобретения, патенты (свидетельства) на
		% полезную модель, патенты на промышленный образец, патенты на селекционные достижения, свидетельства
		% на программу для электронных вычислительных машин, базу данных, топологию интегральных микросхем,
		% зарегистрированные в установленном порядке.(в ред. Постановления Правительства РФ от 21.04.2016 N 335)
	\end{refsection}%
	\begin{refsection}[bl-author, bl-registered]
		% Это refsection=2.
		% Процитированные здесь работы:
		%  * попадают в авторскую библиографию, при usefootcite==0 и стиле `\insertbiblioauthorimportant`.
		%  * ни на что не влияют в противном случае
		\nocite{vakbib2}%vak
		\nocite{patbib1}%patent
		\nocite{progbib1}%program
		\nocite{bib1}%other
		\nocite{confbib1}%conf
	\end{refsection}%
	%
	% Всё, что вне этих двух refsection, это refsection=0,
	%  * для диссертации - это нормальные ссылки, попадающие в обычную библиографию
	%  * для автореферата:
	%     * при usefootcite==0, ссылка корректно сработает только для источника из `external.bib`. Для своих работ --- напечатает "[0]" (и даже Warning не вылезет).
	%     * при usefootcite==1, ссылка сработает нормально. В авторской библиографии будут только процитированные в refsection=0 работы.
}

\end{comment}
% Счётчик \texttt{citeexternal} используется для подсчёта процитированных публикаций;
% \texttt{citeregistered} "--- для подсчёта суммарного количества патентов и программ для ЭВМ.


